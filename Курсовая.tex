\documentclass[oneside, final, 14pt]{extreport}
\usepackage[utf8]{inputenc}
\usepackage[russian]{babel}
\usepackage{vmargin}
\usepackage{amsmath}
\usepackage{graphicx}
\setpapersize{A4}
\setmarginsrb{2cm}{1.5cm}{1cm}{1.5cm}{0pt}{0mm}{0pt}{13mm}
\usepackage{indentfirst}
\sloppy
\begin{document}

\centerline{Московский Государственный Университет им. М. В. Ломоносова}
\centerline{\hfill\hrulefill\hrulefill\hfill}
\vfill
\vfill
\vfill
\large
\vfill
\Large
\begin{centering}
{\bf Исследование алгоритма обнаружения шеллкода PolyUnpack\\}
\end{centering}
\normalsize
\vfill
\centerline{Курсовая работа}
\vfill
\vfill
\begin{flushright}
Выполнил:\\ Платонов И. С. , 322 гр.
\end{flushright}

\begin{flushright}
Научный руководитель:\\
к. ф. м. н., Гамаюнов Д. Ю.
\end{flushright}
\vfill
\vfill
\centerline{Москва--- 2012}








\tableofcontents




\chapter{Аннотация}

Целью курсовой работы является реализация алгоритма PolyUnpack \cite{Poly} по автоматическому нахождению и извлечению обфусцированного и полиморфного кода в исполняемых файлах, с его последующим исследованием. Алгоритм был предложен учеными из Технологического института штата Джорджия, США в 2006 году.  Оригинальная реализация алгоритма отсутствует как в виде исполняемого файла, так и в исходных кодах.

\chapter{Введение}

Большинство современных антивирусных решений используют сигнатурный подход при обнаружении вредоносного программного обеспечения (вирусов, троянов, червей и др. ). В связи с этим, вирусами часто используется механизм обфускации кода программы. Согласно \cite{Obphys} обфускация ("obfuscation" — запутывание) - это один из методов защиты программного кода, который позволяет усложнить процесс реверсивной инженерии кода защищаемого программного продукта. 

Суть процесса обфускации заключается в том, чтобы запутать программный код и устранить большинство логических связей в нем, то есть трансформировать его так, чтобы он был очень труден для изучения и модификации посторонним лицам, таким как взломщики или программисты, которые собираются узнать уникальный алгоритм работы защищаемой программы. На данный момент существует множество методов, применяемых для обфускации кода, основные из них это лексическая обфускация, обфускация данных и обфускация управления. 

Самым распространенным примером лексической обфускации является добавление «мусорных» инструкций и не действующих конструкций, которые не влияют на конечный результат работы программы. Например, все три последовательности не несут никакой полезной нагрузки.
\begin{verbatim}
pushad; nop; nop; nop; nop; popad; 
add eax, 0xffeeffef; inc eax; or eax, eax; sub eax, 0xffeeffee;
push ebx; push ecx; pop ecx; pop ebx; lea eax, [eax]; mov edx, edx;
\end{verbatim}
Еще один вид лексической обфускации - замена блока кода на другой, выполняющий ту же функцию. Например, возможно несколько вариантов записи значения нуль в регистр eax:
\begin{verbatim}
push 0; pop eax;
xor eax, eax;
mov eax, 0;
and eax, 0;
\end{verbatim} 

Средством борьбы с данным видом обфускации является профилировка.

Применяется также перестановка блоков кода программы с сохранением её функциональности:

\begin{verbatim}
00000001    push ecx            00000001    push ecx		 	
00000002    dec ecx             00000002    dec ecx		
00000003    pop ecx             00000003    jmp 0x8		
                                00000007    inc ecx 
                                00000008    pop ecx 
\end{verbatim}

Еще один популярный механизм обфускации, используемый современным ВПО  состоит в том, что код генерируется уже в процессе работы программы, и тут же исполняется. Программа трансформирует или «распаковывает», а затем исполняет блок кода, который был обфусцирован во время компиляции программы. Трансформация, генерирующая код, может быть простой, например применение логической операции XOR между блоком кода и блоком данных, а также достаточно сложной, например, применение некоторого алгоритма шифрования к блоку кода. Пример простой генерации кода изображен на рисунке 2.1. 

\begin{figure}[t]
	\centering
	\includegraphics[width=0.9\textwidth]{Picture52}
	\caption{}
	\label{truck_figure}
\end{figure}

Независимо от сложности применения метода обфускации, простой статический анализатор не распознает в обфусцированном блоке кода инструкции или полностью пропустит анализ этого блока. В работе \cite{Metam} доказана теорема, утверждающая, что проблема распознавания полиморфного кода является в общем случае NP полной. Таким образом, действительное поведение программы будет сокрыто. Конкретными примерами вирусов, генерирующих свой код в процессе исполнения, могут служить шифрованные вирусы, которые сохраняют свой метод шифрования постоянным между копированиями. 

Перед специалистами по информационной безопасности стоит задача разработки моделей детектирования ВПО и методов восстановления операционной системы после заражения. Однако, они часто испытывают сложность, сталкиваясь с примерами самораспаковывающихся (ориг. unpack-execute) образцов, таких как шифрованные и полиморфные вирусы. При их изучении значительное время должно быть потрачено на изучение механизмов, которыми они распаковывают код (в большинстве случаев вредоносный), зашифрованный в процессе компиляции. Только потом он может быть распакован вручную и изучен. 

Основываясь на статистических исследованиях компании «Лаборатория Касперского» \cite{securelist}, каждый день происходят тысячи детектирований вредоносного ПО на компьютерах пользователей, из них несколько сотен являются новыми образцами. Принимая это во внимание, процесс распаковки кода каждого нового образца перед тем, как какой-либо анализ будет осуществлен, может быть затруднительным. Далее силы должны быть потрачены на определение содержит ли образец  самораспаковывающийся код и не являются ли два или более экземпляра одинаковыми с различающимися механизмами обфускации.

\chapter{Описание работы алгоритма}

В данной курсовой работе представлен подход, основанный на анализе поведения программы, который использует комбинацию статического и динамического анализа для автоматизации процесса распаковки скрытого кода, без знания механизма обфускации. Результаты тестирования алгоритмов по поиску вредоносного кода, описанные в статье \cite{h&s} показывают, что подходы, основанные на комбинировании статического и динамического анализа  дают хорошие результаты.

Подход к автоматической распаковке скрытого кода основан на наблюдении, что последовательности распакованного кода в ВПО могут  самоидентифицироваться, когда образец выполняется в среде со знанием статической модели. Для этого должна быть возможность сравнения исполнения программы с последовательностью инструкций, полученной при статическом анализе. Отсутствие такой последовательности в статической модели идентифицирует образец как самораспаковывающийся. Общий обзор всего процесса представлен на рисунке 3.1.

\begin{figure}[t]
	\centering
    \includegraphics[width=0.9\textwidth]{Picture12}
	\caption{}
	\label{truck_figure}
\end{figure}

Анализ каждого нового образца ВПО, мы начинаем с представления статической модели, как должно выглядеть его исполнение, если он не генерирует код и не исполняет его «на лету» (шаг 1). Далее  полученная модель и исследуемый образец поступают на вход динамического анализатора, где образец исполняется в стерильной, изолированной среде. Исполнение останавливается после каждой инструкции и контекст этого исполнения сравнивается со статическим видом модели (шаг 2). Как только выполненная инструкция не находится в статической модели, она выписывается и исполнение останавливается.

Данный подход не определяет, является ли программа вредоносной, а скорее дополняет существующие методы детектирования, которые могут затем использовать распакованный код программы для представления более полного и ускоренного анализа, а также позволяет исследователям вредоносного программного обеспечения не тратить время и силы на его распаковку вручную. 
 
Описанная техника, может предоставлять полный дизассемблированный распакованный код вируса, бинарный дамп кода, или полную исполняемую версию, которая далее может быть загружена в популярные средства анализа, такие как GDB, IDA PRO, или какой-либо алгоритм для поиска шеллкода.

На сегодняшний день отсутствуют утилиты, основанные на аналогичной модели детектирования. Широко распространенная программа Portable Executable (PE) Identifier (PEiD) \cite{peid}, используемая для обнаружения бинарных файлов, которые содержат обфусцированный код, использует метод сопоставления шаблонов. Если находится сопоставление сигнатуры, то знание идентифицированного механизма упаковки может быть использовано для распаковки и извлечения спрятанного кода, содержащегося в бинарных файлах.

Главные ограничения PEiD идентичны тем, которые содержат антивирусные средства, использующие сигнатурный подход. База сигнатур должна постоянно обновляться для детектирования новых образцов.  Таким образом, утилита может потерпеть неудачу даже при незначительных изменениях в  известном механизме упаковки, если новая сигнатура не будет заблаговременно включена в базу.

Ближайшая индустриальная работа --- это Universal PE Unpacker plugin,  доступный к IDA Pro 4.9 \cite{ida}. Плагин использует эвристику, основанную на поведении, которое заключается в том, что программа возвращается в точку входа после того, как она стала распаковываться. Хотя это и хороший подход для детектирования некоторых техник упаковки, здесь присутствует немалая вероятность уклонения. Предположим, что программа, начинающая свое выполнение с точки входа, трансформирует порцию данных в одной из своих секций, затем направляет исполнение в эту секцию. Этот прием обойдет технику детектирования плагина. 

Такие утилиты как PE-Explorer \cite{peexplorer} позволяют добавлять в исполняемый файл новые секции кода, редактировать уже имеющиеся и изменять права доступа к ним. Так что уклонение от такой эвристики не составляет труда. Описанный в данной работе подход не делает аналогичное допущение, а предлагает более общее решение проблемы распаковки обфусцированного кода.

\section{Формализация процесса самораспаковки}

Идея автоматизации процесса распаковки происходит из наблюдения, что программы пытаются скрыть свое поведение через обфускацию, потому что код, который они прячут вредоносный. То есть, если бы код в обфусцированном блоке был доступен для статического анализатора,  программа была бы распознана как вредоносная. Таким образом, код, содержащий во время компиляции обфусцированный блок не появится в какой-либо последовательности инструкций при статическом анализе. Это наблюдение дает подход к формальному определению поведения, обобщающего одновременно зашифрованные и полиморфные вирусы и метода для автоматизации процесса распаковки их спрятанного кода, не зависимо от используемых алгоритмов распаковки. 

%\section{Простое определение программы}

В современных компьютерах, программы могут быть представлены как множество,  состоящее из двух упорядоченных множеств данных I и D, то есть из двух кортежей. Кортеж \(I=(i_0,\ldots , i_n)\) представляет набор упорядоченных последовательностей, представимых как  инструкции, где \(i_k=(o_0,\ldots , o_j) , 0<k<n\) --- особенный упорядоченный набор инструкций и \(o \in i_k\) ---  инструкция в этой последовательности. Примером \(o_i\) является опкод \begin{verbatim} ffffff8bffffffc3, \end{verbatim} представимый в виде ассемблерной инструкции \begin{verbatim} mov eax, ebx. \end{verbatim} Аналогично, кортеж \(D=(d_0,\ldots , d_m)\) --- это определенный набор упорядоченных последовательностей данных, но не представимых как инструкции, например db 0xffffffd9 . Таким образом, программа P=(I,D). Пример части программы показан на рисунке 3.2. 

\begin{figure}[t]
	\centering
	\includegraphics[width=0.9\textwidth]{Picture22}
	\caption{}
	\label{truck_figure}
\end{figure}

Под обычным исполнением программы P  понимают исполнение, где P не генерирует и не исполняет новые последовательности инструкций. В любой момент времени обычного исполнения специальный регистр eip (extended instruction pointer), будет указывать на некоторую инструкцию \(o \in i, i \in I\). Начиная с этой инструкции o и инкрементируя eip t раз (0<t) будет воспроизведена упорядоченная последовательность \(i_t=(o_0,\ldots ,o_t)\), где \(i_t\), некоторая подпоследовательность одной или нескольких \(i \in I\).

\section{Трансформация данных во время исполнения} 

В процессе исполнения, P может иметь доступ к данным вне своих собственных данных D. Например P может использовать пользовательский ввод, читать файлы с диска, открывать сетевые соединения и др. Обозначим внешние данные как \((e_0,\ldots, e_l)\). Пусть кортеж \(E=(d_0,\ldots , d_m, e_0, \dots ,e_l)\) --- упорядоченный набор данных, к которым программа P может иметь доступ, когда она загружается и исполняется. Упорядоченная последовательность инструкций \(i_p\), состоящая из возможно повторяющихся, упорядоченных подпоследовательностей инструкций из членов принадлежащих множеству I могут представлять функцию \(\pi _p\). Обозначая вызов \(\pi _p\) в исполняющейся программе P как \(\pi _p(E)\), функция \(\pi _p\), использует члены E для трансформации данных \(d _p\).

%\section{Определение поведения самораспаковывающихся программ}

При обычном исполнении программы P, результат функции трансформации данных \(d_p\) может рассматриваться только как данные, даже если она представляет легитимные упорядоченные последовательности инструкций.  
Однако, представляется возможность, что \(d_p\) может быть подана на исполнение программы. Это может происходить либо из инкрементирования eip до того момента, пока он не укажет на инструкцию \(o \in d_p\), либо после исполнения функции перехода, адрес которой указывает на o. Данная ситуация показана на рисунке 3.4.

\begin{figure}[t]
	\centering
	\includegraphics[width=0.9\textwidth]{Picture32}
	\caption{}
	\label{truck_figure}
\end{figure}

\begin{figure}[t]
	\centering
	\includegraphics[width=0.9\textwidth]{Picture42}
	\caption{}
	\label{truck_figure}
\end{figure}

Последовательность сгенерированных инструкций в процессе исполнения остается просто данными, если она никогда не выполняется. Таким образом, возможно введение определения поведения самораспаковывающейся программы. Оно охватывает самораспаковывающийся и полиморфный классы ВПО. Полиморфный вирус может менять метод криптования при копировании, но подобная мутация не влияет на его поведение.

Программа P называется самораспаковывающейся в некоторой точке в процессе своего выполнения, если eip указывает на инструкцию о в исполняющейся последовательности инструкций \(i_p\), где для любого \(i \in I,\ i_p\) не является подпоследовательностью \(i\).

\section{Описание работы}

Алгоритм распаковки и извлечения кода, описанный ниже, использует статический вид кода в программе P, поданной ему на вход, как модель для сравнения, когда происходит одиночное пошаговое исполнение. Это служит техникой распознавания, проявляет ли данная программа поведение, удовлетворяющее условиям определения.

Возможна ситуация, когда неизвестный образец вируса может не проявлять самораспаковывающегося поведения, в том смысле что код, сгенерированный в процессе работы программы не будет исполняться. Для решения данной проблемы был введен дополнительный входной параметр kol\_ step, который определяет количество инструкций, которые будут исполнены до остановки выполнения программы. Таким образом, исполнение программы будет остановлено 
\\1) при достижении максимального допустимого количества шагов \\ 2) регистр eip указывает на инструкцию вне программы \\ 3) eip указывает на конец буфера

Входные параметры: Входная программа P и граница исполнения программы kol\_ step.
Выходные параметры: Выходная последовательность инструкций \(i_p\), представляющая код, полученный в процессе исполнения, либо NULL, если P была остановлена без самораспаковки или граница исполнения инструкций kol\_ step была достигнута.
\begin{verbatim}
begin{
    //шаг1: Статический анализ
    //Дизассемблирование блока кода программы P для определения инструкций и данных. Все полученные инструкции заносятся в список с указанием отступа от начала файла.
    I=Disassemble(P)
    //Шаг 2: Динамический анализ. Исполняем программу P по одной 
    //инструкции. На каждом шаге получаем выполненный опкод, 
    //дизассемблируем его и сравниваем с текущей инструкцией в 
    //списке I. Если они не совпадают, то делаем вывод, что был
    // исполнен код, полученный в процессе выполнения и выписываем его.
    for (int i=0; i<kol_step; i++){
        1. Исполняем инструкцию, на которую указывает регистр eip.
        2. Представляем исполненный опкод в дизассемблированном виде.
        3. Сравниваем полученную инструкцию с инструкцией из статического анализа. Если они совпадают, то переходим на шаг 1, иначе на шаг 4.
        4.Выписываем полученную инструкцию.
    }
}
\end{verbatim}
Данный алгоритм был реализован под ОС Ubuntu 10.04 семейства Unix. Для статического анализа входной программы используется библиотека для дизассемблирования опкодов в формате  Intel x86 libdasm \cite{libdasm}, реализованная на языке C. Далее для динамического выполнения программы покомандно была выбрана библиотека  libemu \cite{libemu}, позволяющая производить эмуляцию и поиск шеллкода. Обычно она используется при разработке IDS(Систем Обнаружения Вторжений), IPS(Систем Предотвращения Вторжений ) и honeypots, используемые для обнаружения ботнетов. 

Использование данной библиотеки, во-первых, позволяет значительно сократить время, необходимое для обработки каждого образца, по сравнению с использованием уже готовых утилит для по-командного исполнения программы. Во-вторых, она дает возможность не пользоваться различными виртуальными машинами для запуска вирусов, а запускать вирусы на своей основной, рабочей операционной системе.

\section{Множественная обфускация}

Некоторые образцы самораспаковывающегося ВПО усложняют процесс извлечения кода путем многократной распаковки. Возможна ситуация, когда последовательность инструкций, появившаяся в процессе исполнения программы, так же будет обфусцирована, то есть некоторый блок кода содержит множественную обфускацию. Для такого рода вирусов, алгоритм может быть использован для получения полностью распакованного кода. Для этого после первого исполнениея kol\_ step инструкций и обнаружения обфускации кода, необходимо запомнить всю секцию кода. После этого повторить исполнение kol\_ step инструкций еще раз, на тех же данных, начиная с точки входа. Далее сравнить буфер, полученный после повторного исполнения с буфером, полученным после однократного. Если они отличаются, то можно утверждать, что была применена множественная обфускация. Тогда необходимо повторять процесс исполнения программы, начиная с точки входа столько раз, пока буфер после  \(i-1\)-го исполнения kol\_ step команд не будет совпадать с буфером после \(i\)-го исполнения. Полученный таким образом код будет являться полностью распакованным и не обфусцированным. 

\section{Уклонение}
У данного метода есть свои особенности.

В первую очередь динамический анализ покрывает программу не полностью. Метод динамического анализа предлагает лишь малую часть из возможных вариантов исполнения программы. Более того, множество существенных вариантов работы программы не могут быть детектированы. 

Также в некоторых вирусах содержатся специальные техники, способные проверять на исполнение в виртуальной среде. Например, в Win32 API содержится функция bool IsDebuggerPresent(). Также программа может следить за системным временем. Если между исполнением двух соседних инструкций проходит длительный интервал времени, возможен вывод, что её отлаживают. В таких случаях вирус может изменить свое поведение, для того, чтобы скрыть свои вредоносные свойства. 

В некоторых случаях возможно попадание эмулятора в бесконечные циклы. Они намеренно вставляются в программный код перед декодедором. В таком случае его обнаружение не осуществится. В таких ситуациях может помочь техника, описанная в \cite{loop}.

\chapter{Результаты}

Реализация данного алгоритма доступна по адресу

В начале испытания проводились на обфусцированных файлах в формате *.raw. Тестовая выборка из 60 файлов была получена с помощью платформы Metasploit Framework \cite{meta}, позволяющая отлаживать и создавать эксплойты, а также предоставляющая обширную базу опкодов и шеллкодов. После этого файлы были обфусцированны с помощью представленных в ней механизмов. Формат *.raw является наиболее подходящим для испытаний, поскольку в перспективе планируется работа реализованного алгоритма на сетевом трафике. На некоторых образцах процесс распаковки кода занимал до 2250 шагов. Таким образом была получена нижняя оценка в размере 2250 инструкций, которые должны исполниться до остановки программы. Были получены положительные результаты, а именно была распознана обфускация во всех имеющихся файлах. К сожалению, данная платформа предоставляет лишь 12 различных механизмов обфускации. Вследствие чего было принято решение ввести поддержку формата Portable Executable. 

Для тестирования алгоритма была собрана выборка из 1500 валидных файлов формата Portable Executable, что составляет около 1,6 Гб. Количество ложных срабатываний (False Positives) во время работы не привысило 2\%. 
Полные результаты работы реализованного алгоритма представлены в 
таблице 4.1. Аппартные характеристики машины, на которой производилось тестирование: Intel Pentium Dual-Core T4400 2.2 ГГц, 2048 RAM . Исходя из этих данных, можно заключить, что возможна корелляция количества шагов от политики безопасности и требований, предъявляемых к производительности, но проверять больше шагов, чем 5000 не имеет смысла.

\begin{table}[t]
\begin{tabular}{|r|c|p{0.15\textwidth}|p{0.25\textwidth}|c|r}
\hline & \multicolumn {4}{c|}{Количество инструкций}\\
\hline & 2300 & 3500 & 5000 & 10000\\
\hline Время, сек & 356 & 475 & 588 & 862\\
\hline True Positives & 1255 & 1251 & 1249 & 1249\\ 
\hline False Positives & 19 & 23 & 25 & 25\\ 
\hline Скорость, байт/c & 4706 & 3528 & 2850 & 1944\\ \hline
\end{tabular}
\caption{Результаты работы на валидной выборке }
\end{table}

В проведенном испытании ложные срабатывания происходили из-за того, что исполнялись файлы формата, не поддерживаемого OC. При эмуляции исполнения переходы во внешние функции вида \begin{verbatim} call [0x402030] \end{verbatim} или \begin{verbatim} jmp [0x402030]\end{verbatim} игнорировались. В связи с этим значения регистров могли быть другими. Соответственно вызовы функций вида call reg или команды перехода jmp reg могли вести в область вне загруженной программы, что и вызывало ложные срабатывания. 

Также было получено около 1300 образцов вредоносных файлов из репозитория OARC, что составляет около 350 Мб. Содержание данного репозитория полу-открытое и доступно для специалистов по информационной безопасности по запросу. Образцы, содержащиеся в этой выборке были детектированы в период с  по  . Все файлы в ней уникальны, о чем свидетельствуют результаты применения к ним хэш функции MD5. 

Затруднение вызывает то, что не существует точной статистики по количеству обфусцированных файлов как в данной выборке, так и в принципе. Было принято решение сверять каждое детектирование с результатом работы утилиты PEiD, упоминавшейся ранее. Как было сказано ранее, PEiD — это популярная утилита для реверс инжиниринга, которая использует базу сигнатур для определения, является ли код обфусцированным. Она не распаковывает его, а лишь определяет какой алгоритм обфускации был применен.  Полные результаты работы алгоритма на данной выборке представлены в таблице 4.2. Обфускация была распознана во всех файлах, которые были отмечены PEiD как обфусцированные, а также в некоторых, которые утилита пропустила. Это свидетельствует о том, что что алгоритм действительно способен находить обфускацию кода в бинарных файлах без знания механизма упаковки. 

\begin{table}[t]
\begin{tabular}{|r|c|p{0.15\textwidth}|p{0.25\textwidth}|c|r}
\hline & \multicolumn {4}{c|}{Количество инструкций}\\
\hline & 2300 & 3500 &5000 & 10000\\
\hline Время, сек & 305 & 387 & 448 & 819\\
\hline True  Negatives & 494 & 498 & 501 & 501\\ 
\hline False Negatives & 714 & 710 & 707 & 707\\ 
\hline Скорость, байт/c & 1014 & 800 & 690 & 377 \\ \hline
\end{tabular}
\caption{Результаты работы на выборке из вирусов}
\end{table}
  
 
\section{Время исполнения}

Как можно заметить из сравнения таблиц 4.1 и 4.2, скорость работы на выборке из вирусов значительно меньше. Принимая во внимания тот факт, что количество файлов в обеих выборках примерно одинаковое (1274 и 1208 соответственно), это может быть связано в первую очередь с использованием медленных операций работы с файлом. Также на время выполнения влияет то, что после обнаружения первой обфусцированной инструкции, указатель на следующую инструкцию в списке, полученном при статическом дизассемблировании, становится NULL. После каждого шага, происходит попытка найти место для указателя. Таким образом после каждого шага просматривается в среднем половина списка инструкций.

По статистике, распаковка одного обфусцированного файла занимает от 15 до 60 минут. В ситуации, сложившейся на сегодняшний день, когда ежедневно появляется несколько сот новых образцов, ясно, что необходимо создание эффективного средства, автоматизирующего процесс распаковки. Из таблицы видно, что средняя скорость работы на валидной выборке при исполнении 3500 инструкций, составляет 3530 байт в секунду , а на не валидной  800 байт в секунду. Следовательно, реализованный алгоритм может служить подобным средством, не уменьшая точности последующего анализа файла.   

\chapter{Заключение}
Результаты работы реализованного алгоритма показывают, что скорость его работы и точность удовлетворяют требованиям, предъявляемым утилитам, которые работают с сетевым трафиком. Таким образом, в перспективе возможно встраивание реализации данного алгоритма в систему обнаружения вторжений RedSecure. Поскольку реализация данного алгоритма покрывает только класс обфусцированных вирусов, то его использование возможно только в качестве дополнительного анализатора к другим алгоритмам детектирования шеллкода.

\begin{thebibliography}{00}
\bibitem{Poly} Paul Royal, Mitch Halpin, etc. \emph{ PolyUnpack: Automating the Hidden-Code Extraction of Unpack -Executing Malware.} College of Computing Georgia Institute of Technology, ACSAC, 2006
\bibitem{Obphys} Варновский Н.П., Захаров В.А \emph{Современные методы обфускации программ: сравнительный анализ и классификация}. Московский Государственный университет, Известия ЮФУ, 2007 
\bibitem{Metam} E. Filiol\emph{Metamoorphism, formal grammars and undecidable code mutation}. International Journal of Computer Science, 2007
\bibitem{h&s} Dennis Gamayunov, Svetlana Gaivoronovski. \emph{Hide and Seek:  worms digging at the Internet backbones and edges}. Computer Science Dept. of Moscow State University, 2010
\bibitem{shellcode} Chris Anley, John Heasman, etc. \emph{The shellcoder's handbook:Discovering and Exploiting Security Holes}. Wiley, 2007
\bibitem{network} M. Polychronakis, K. G. Anagnostakis, etc. \emph{Network-level polymorphic shellcode detection using emulation.} Proceedings of the Conference on Detection of Intrusions and Malware \& Vulnerability Assessment. Berlin: Springer-Verlag, 2006
\bibitem{loop} Tubella, J., Gonzalez, A. \emph{Control speculation in multithreaded processors through dynamic loop detection.} Proceedings of the 4th International Symposium on HighPerformance Computer Architecture (HPCA), 1998
\bibitem{market} Team Cymru \emph{Malware Infections Market.} http://www.team-cymru.com/ReadingRoom/Whitepapers/2010/Malware-Infections-Market.pdf
\bibitem{Stolyarov} Столяров А.В. \emph{Программирование на языке ассемблера Nasm для OC Unix}. Макс Пресс, 2010
\bibitem{Yurov} Юров В.И. \emph{ Ассемблер. Учебник для вузов}. изд. Питер, 2005
\bibitem{PE} \emph{Microsoft  PE and COFF Specification}  http://msdn.microsoft.com/en-us/windows/hardware/gg463119
\bibitem{securelist} http://www.securelist.com/en/analysis/204792216/Kaspersky\_Security\_Bulletin\_Statistics\_2011
\bibitem{peid} http://www.peid.info/
\bibitem{meta} http://www.metasploit.com/
\bibitem{ida} http://www.hex-rays.com/products/ida/index.html
\bibitem{peexplorer} http://www.heaventools.com/overview.html 
\bibitem{libemu} http://libemu.carnivore.it/
\bibitem{libdasm} http://code.google.com/p/libdasm/
\end{thebibliography}

\end{document}